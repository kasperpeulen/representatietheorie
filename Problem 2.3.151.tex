\thm{2.3.16a}{Let $A$ be an algebra over a field $k$. The center $Z(A)$ of $A$ is the set of all elements $z \in A$ which commute with all elements of $A$. For example, if $A$ is commutative, then $Z(A)=A$. Show that if $V$ is an irreducible finite dimensional representation of $A$, then any element $z \in Z(A)$ acts in $V$ by multiplication by some scalar $\chi_V : Z(A) \to k$ is a homomorphism. It is called the central character of $V$.}

As $V$ is representation of $A$, there is some action such that $a.v\in V$ for $a \in A$ and $v\in V$. We need to show that if $a \in Z(V)$, then $a.v = k_1 v$ for some $k_1 \in k$ for all $v\in V$. 

So we need to show that there exists a $k_1$ such that $a.v=k_1.v$. 
Or equivalently, $(a-k_1).v=0$, for all $v \in V$. 
So we need to show that there exists  a $k_1$ such that $\rho(a)=k_1I$.

In other words, we need to show that $\rho(a)$ is a scalar matrix, for any $a \in Z(A)$. 
We define $\rho(a)=A$, and we know that $AB=BA$ for every $B=\rho(b)$ where $b\in A$. 