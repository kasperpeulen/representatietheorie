\opgave{7g}{Explain that a simple ring $R$ is always an algebra over some field.}
Suppose that we would know that center of $R$ is a field. Well then we've got a ringhomomorphism 
\[Z(R) \to Z(R) : r \mapsto r .\]
\par And by definition this makes $R$ an algebra. So we can prove the proposition by proving the following lemma. 

\theorem{Lemma 1}{The center of a simple ring is a field.}
Suppose $a\in Z(R)-{0}$. Then $aR$ is a non-zero two-sided ideal of $R$ and since $R$ is simple, $aR$ must be equal to $R$. And since $1 \in R$, it should be possible to write $1 = aa^*$ for some $a^* \in R$. If we can show that this $a^*$ is in $Z(R)$, then we are done. So we need to show that $ra^*=a^*r$, or equivantly, $ra^*a=a^*ra$. This surely holds as $a\in Z(R)$. 

